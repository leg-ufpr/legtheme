\documentclass[a4paper,$if(font-size)$$font-size$$endif$]{article}

\usepackage[brazilian]{babel}
\usepackage[T1]{fontenc}
\usepackage[utf8]{inputenc}
\usepackage{amsmath,amsfonts,amssymb,amsthm}
\usepackage[margin=2.5cm]{geometry}
\usepackage{indentfirst}
\usepackage{graphicx}
%\usepackage{subfigure}
\usepackage{url}
\usepackage{supertabular}
\usepackage{lscape}
\usepackage{setspace} % espaçamento 1,5
\usepackage[font={small}]{caption}

\usepackage{titlesec}
\titleformat*{\section}{\Large\bfseries\sffamily}
\titleformat*{\subsection}{\large\bfseries\sffamily}
\titleformat*{\subsubsection}{\bfseries\sffamily}

%% Fontes
\usepackage[scaled]{helvet}
% \usepackage{times,mathptmx}
\usepackage[sc]{mathpazo}
\usepackage{inconsolata}


$if(highlighting-macros)$
$highlighting-macros$
$endif$

\begin{document}
%\maketitle
% a inclusão desse comando depende do que foi especificado acima
% \thispagestyle{fancy}

\singlespacing

\pagestyle{empty}

\begin{center}

\large
\textbf{$if(university)$$university$$endif$}\\
\textbf{$if(sector)$$sector$$endif$}\\
\textbf{$if(department)$$department$$endif$}

\vspace{4cm}
%\large
$if(author)$
$author$
$endif$

\vspace{5cm}
\Large
\textbf{$if(title)$$title$$endif$}

\vspace{\stretch{1}}
\large
$if(city)$$city$$endif$\\
$if(year)$$year$$endif$

\end{center}

\pagebreak

\clearpage\mbox{}\clearpage

\normalsize
\onehalfspacing

$if(toc)$
\pagebreak
\tableofcontents
$endif$

\pagebreak

\pagestyle{plain} % headings (igual a book)
\pagenumbering{arabic}

$body$

\end{document}
